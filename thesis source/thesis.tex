%
% IMPORTANT:
% this document contains packages and styles, the content of this thesis is located in file thesis-content.tex
%

\documentclass[
a4paper, 11pt,
toc=bibliography,
chapterprefix=true, numbers=noenddot,% prints ``Chapter 1'' instead of just ``1.''
%page formatting, BCOR-> binding correction
DIV=15,BCOR=4mm,headinclude=true,footinclude=false
]{scrbook}

\areaset[12mm]{469pt}{710pt}

%returns the current file name/structure
\usepackage{currfile}
%to allow to generate pdf files for subsections of this thesis
\usepackage{subfiles}

%needed for prettier tables
\usepackage{tabularx,booktabs}
%for merged cells in tables
\usepackage{multirow,multicol}

%generates dummy text
\usepackage{lipsum}
%supplies text colors and text highlighting, underlining,...
\usepackage{xcolor, soul}

%needed to include images
\usepackage{graphicx}

%used in the appendix to include pdf pages
\usepackage{pdfpages}

\usepackage[utf8]{inputenc}
\usepackage[T1]{fontenc}
\usepackage{lmodern}

\usepackage[bookmarks]{hyperref}

%for bibtex citation
%\usepackage[backend=bibtex,style=alphabetic]{biblatex}
\usepackage[backend=bibtex,style=ieee,urldate=comp]{biblatex}
\bibliography{websources,rfc,sources}

%language in document
\usepackage[ngerman, english]{babel}
\selectlanguage{english}


%this is a shorthand to citing RFCs
\newcommand{\rfc}[1]{{\cite{rfc#1}}}

%this is used to write comments in the text, that will not be displayed
\long\def\comment #1\endcomment{}
%this is used to highlight todo items
\newcommand{\todo}[1]{{\LARGE\textbf{\hl{#1!!}}}}

%allows to use figures with multi-line captions
\usepackage[margin=0.1\textwidth,font=small,labelfont=bf,singlelinecheck=false]{caption}

\usepackage{xparse}
%this allows to insert an image like this:
%\img[short caption for tableoffigures]{filename in \currfolder/res/}{caption}
%or
%\img{filename in \currfolder/res/}{caption}
\DeclareDocumentCommand \img { o m m } {%
	\begin{figure}[h]
		\centering
		{\includegraphics[width=0.9\textwidth,keepaspectratio]{\currfolder/res/{#2}}}%
		\IfNoValueTF{#1}%
			{\caption{#3}}%
			{\caption[#1]{#3}}%
		\label{fig:#2}
  	\end{figure}
}

%specify heigh as the fourth parameter
\DeclareDocumentCommand \imgWithHeight { o m m m } {%
	\begin{figure}[h]
		\centering
		\includegraphics[width=0.9\textwidth,height={#4},keepaspectratio]{\currfolder/res/{#2}}%
		\captionsetup{singlelinecheck=on}%if image width is smaller than caption, center caption

		\IfNoValueTF{#1}%
			{\caption{#3}}%
			{\caption[#1]{#3}}%
		\label{fig:#2}
  	\end{figure}
}


\newcommand*{\captionsource}[2]{%
  \caption[{#1}]{%
    {#1}%
    \\\hspace{\textwidth}%
    \textbf{Source:} {#2}%
  }%
}


\DeclareDocumentCommand \imgWithSource { m m m } {%
	\begin{figure}[h]
		\centering
		\includegraphics[width=0.9\textwidth]{\currfolder/res/{#1}}
		\captionsource{{#2}}{{#3}}
		\label{fig:#1}
  	\end{figure}
}

\DeclareDocumentCommand \imgWithSourceAndHeight { m m m m } {%
	\begin{figure}[h]
		\centering
		\includegraphics[width=0.9\textwidth,height={#4},keepaspectratio]{\currfolder/res/{#1}}
		\captionsource{{#2}}{{#3}}
		\label{fig:#1}
  	\end{figure}
}


%needed for url display
\usepackage{url}

%used to display the glossary from file glossary.tex
\usepackage{calc}


%header and footer styles
\usepackage[headsepline, automark]{scrlayer-scrpage}

\clearpairofpagestyles 
\ihead{\headmark} 
\ohead[\pagemark]{\pagemark}
%\chead{\automark*[part]{} ppp}
%\ofoot[\pagemark]{\pagemark}

%end header and footer styles

%chapter styles
\renewcommand{\chapterheadstartvskip}{\vspace*{-.5\topskip}}
\addtokomafont{chapterprefix}{\raggedleft}
%\addtokomafont{chapter}{\fontsize{30}{38}\selectfont}

\renewcommand*{\chapterformat}{%
\underline{
\color{gray}\mbox{%
%\scalebox{1.5}{\color{black}\raisebox{.7ex}{\chapappifchapterprefix{}}}%
\scalebox{1.5}{\color{black}{\chapappifchapterprefix{\nobreakspace}}}%
%\scalebox{1.5}{\color{black}\raisebox{.7ex}{\chapappifchapterprefix{\nobreakspace}}}%
%\scalebox{3}{\color{gray}\thechapter\autodot\rule[1pt]{.3pt}{.55em}}\enskip}%} %number
\scalebox{3}{\color{gray}\thechapter\autodot}\enskip}%} %number
}
}
%end chapter styles

%command for code listings
\newcommand{\code}[2]{\lstinputlisting[language=JavaScript,label={lst:#1},#2]{\currfolder/res/#1}}
%command

%styles for code listings
\usepackage{listings}
%\definecolor{lightgray}{rgb}{.9,.9,.9}
\definecolor{darkblue}{rgb}{.2,.2,.5}
\definecolor{olive}{rgb}{.25, .52, .08}
\definecolor{purple}{rgb}{0.65, 0.12, 0.82}
\lstdefinelanguage{JavaScript}{
  keywords={break, case, catch, continue, debugger, default, delete, do, else, finally, for, function, if, in, instanceof, new, return, switch, this, throw, try, typeof, var, void, while, with},
  morecomment=[l]{//},
  morecomment=[s]{/*}{*/},
  morestring=[b]',
  morestring=[b]",
  ndkeywords={class, export, true, false, implements, import, null this},
  keywordstyle=\color{blue}\bfseries,
  ndkeywordstyle=\color{darkblue}\bfseries,
  identifierstyle=\color{black},
  commentstyle=\color{purple}\ttfamily,
  stringstyle=\color{olive}\ttfamily,
  sensitive=true,
  frame=tb,
  framesep=5pt,
  framexleftmargin=2em, %include line numbers in frame
  framexrightmargin=.5em,
  xleftmargin=20mm,
  xrightmargin=10mm,
  aboveskip=1em%,  belowskip=.5em
}

\lstset{
   language=JavaScript,
   extendedchars=true,
   basicstyle=\footnotesize\ttfamily,
   showstringspaces=false,
   showspaces=false,
   numbers=left,
   numberstyle=\footnotesize,
   numbersep=9pt,
   tabsize=2,
   breaklines=true,
   showtabs=false,
   captionpos=b
}
%end styles for code listings



\begin{document}


%to test citation styles
\comment
%\endcomment
\chapter{test citations}
abc\cite{Doyle.2001}

using rfc command\rfc{5766}
adfa\cite{rfc5766}
another rfc(without shorthand)\cite{rfc2}
rfc001\cite{rfc1}
an internet page citation\cite{Bry2005}
a missing bibtex entry to an internet page\cite{thz_ne}
\comment
\endcomment

%test code listing style
\comment
\begin{lstlisting}[caption=My Javascript Example]
Name.prototype = {
  methodName: function(params){
    var doubleQuoteString = "some text";
    var singleQuoteString = 'some more text';
    // this is a comment
    if(this.confirmed != null && typeof(this.confirmed) == Boolean && this.confirmed == true){
      document.createElement('h3');
      $('#system').append("This looks great");
      return false;
    } else {
      throw new Error;
    }
  }
}
\end{lstlisting}
\endcomment


%includes all needed files without adding a page break
\frontmatter

\title{WebRTC multipoint conferencing}
\subtitle{with recording using a Media Server}
\author{Marc Walter}
\date{February 26, 2015 }

%create @author and @date values for the document
\makeatletter

\begin{titlepage}

	\vspace{\fill}
	\vspace{\fill}

	%\flushleft{\large \@author}

	\vspace{10mm}

	\centering\textbf{\Huge \@title\\ \@subtitle}\\

	\vspace{\fill}

	\centering 
		\@author\\
		Hochschule der Medien Stuttgart\\
		Stuttgart Media University\\
		Bachelor's Thesis

	\vspace{\fill}

	\begin{flushleft}
		%table with different information
		\begin{tabular}{llll}
		\textbf{Author:} & & \@author & \\
		& & Walter.Marc@outlook.com & \\
		& & MatNr. 24650 & \\
		& & \\
		\textbf{Course:} & & Bachelor Medieninformatik\\
		& & \\
		\textbf{Date:} & & \@date &\\
		& & \\
		\textbf{Supervisors:} & & Prof. Walter Kriha, HdM Stuttgart &\\
		& & Dipl.Inf.(FH) Matthias Litz, BeamYourScreen GmbH &\\
		\end{tabular}
	\end{flushleft}

\end{titlepage}

% start a group that keeps changes local
\begingroup
\let\clearpage\relax %disable \clearpage
\thispagestyle{empty} %no header and footer
\parindent0pt %deactivate indentation

\begin{tabularx}{\textwidth}{ XX XX }
\textbf{Name:} & Walter & \textbf{Vorname:} &Marc\\
\textbf{Matrikel-Nr:} & 24650 & \textbf{Studiengang:} & Medieninformatik\\
\end{tabularx}
\vspace{6mm}
\hrule



\vspace{20mm}

\textbf{\Large{Eidesstattliche Versicherung}}\\

Hiermit versichere ich, Marc Walter, an Eides statt, dass ich die vorliegende Bachelorarbeit mit dem Titel: \textbf{WebRTC multipoint conferencing with recording using a Media Server} selbstständig und ohne fremde Hilfe verfasst und keine anderen als die angegebenen Hilfsmittel benutzt habe. \\
Die Stellen der Arbeit, die dem Wortlaut oder dem Sinn nach anderen Werken entnommen wurden, sind in jedem Fall unter Angabe der Quelle kenntlich gemacht. Die Arbeit ist noch nicht veröffentlicht oder in anderer Form als Prüfungsleistung vorgelegt worden.\\

Ich habe die Bedeutung der eidesstattlichen Versicherung und die prüfungsrechtlichen Folgen (§26 Abs. 2 Bachelor-SPO (6 Semester), § 23 Abs. 2 Bachelor-SPO (7 Semester) bzw. § 19 Abs. 2 Master-SPO der HdM) sowie die strafrechtlichen Folgen (gem. § 156 StGB) einer unrichtigen oder unvollständigen eidesstattlichen Versicherung zur Kenntnis genommen.


\vspace{15mm}
Stuttgart, den 23.02.2015


\vspace{3cm}
Marc Walter


% end group
\endgroup


\clearpage
% start a group that keeps changes local and disable \clearpage
\begingroup\let\clearpage\relax
\thispagestyle{empty} %no header and footer


\noindent{\sffamily \textbf{\huge{Abstract}}}\\\\
\noindent
Goal is to both research and implement a multipoint communication solution that uses WebRTC for audio and video transmission. The conference size should be arbitrary but support at least four participants and allow multiple conferences at the same time. 
The solution should contain recording functionality and allow participation of devices in restrictive network environments.\\

These goals were met by using a a media server that allows to significantly increase the conference size from the maximum of four when using a full mesh peer-to-peer architecture.\\

This thesis is separated into two parts: The first consists of an introduction to WebRTC, research and theoretical assumptions that lead to an implementation strategy. The second part contains information about the selected media server, follows the implementation of the prototype solution and describes parts of its architecture.\\
The prototype itself is not part of this thesis.\\

Included are benchmark results for WebRTC conferences concerning the bandwidth requirements for different multipoint architecture patterns and screen resolutions. Those were conducted using the prototype application.\par



\vspace{3cm}

\selectlanguage{ngerman}% ensure the right hyphenation patterns for German
\noindent{\sffamily \textbf{\huge{Zusammenfassung}}}\\\\
\noindent 
Ausgangspunkt für diese Thesis ist das Ziel, eine auf WebRTC basierende Web\-anwendung als Kommunikationslösung zu schaffen, die mindestens vier Teilnehmer in einer Konferenz kommunizieren lässt, mehrere Konferenzen gleichzeitig unterstützt, die Aufnahme der Konferenzen ermöglicht und bei der auch Teilnehmer in restriktiven Netzwerkumgebungen teilnehmen können. Dieses Ziel wurde in vollem Umfang erreicht.\\

Hierbei wird ein Media Server eingesetzt, um die maximale Teilnehmeranzahl bei Konferenzen im Vergleich zu direkten Peer-To-Peer Verbindungen, die bei vier Teilnehmern je nach Verbindungsgeschwindigkeit an ihre Grenzen stoßen, fast beliebig zu vergrößern.\\

Nach einer Einführung in WebRTC und Vorüberlegungen für die gewünschte Kommunikationslösung im ersten Teil dieser Thesis werden im zweiten Teil die Auswahl eines Media Servers sowie der Prototyp und dessen Architektur beschrieben.\\
Diese Prototypanwendung ist jedoch nicht Teil dieser Thesis.\\

Des Weiteren wurden mit der Prototyp-Anwendung Vergleiche zu Bandbreitenanforderungen von WebRTC bei der Verwendung von verschiedenen Kommunikationsarchitekturen und Bildgrößen der Videoaufnahmen durchgeführt und dokumentiert.\par

\vspace{\fill}

% end group
\endgroup

\tableofcontents

\mainmatter
\comment
\endcomment
\edef\folder{subdocuments}

%\mainmatter\setcounter{page}{1}
%part 1: Theory
\subfile{\folder/part1}

%part2: practical stuff
\subfile{\folder/part2}


\appendix
\part{Appendix}
\subfile{\folder/appendix}


\addcontentsline{toc}{chapter}{\listfigurename}
\listoffigures

\addcontentsline{toc}{chapter}{\listtablename}
\listoftables

\renewcommand{\lstlistingname}{List of Code Listings}
\addcontentsline{toc}{chapter}{\lstlistingname}
\lstlistoflistings

%include manually created glossary file
\subfile{\folder/glossary}


%uses biblatex
\printbibliography

\comment
\endcomment



\end{document}