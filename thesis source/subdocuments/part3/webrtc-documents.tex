\addchap{WebRTC documents}%no number
\label{chap:webrtc-documents}

\section{Overview}
Aim of the WebRTC projects is to enable real time communication in a web browser without needing plugins, platform dependent applications or specific servers.
The browsers communicate and transmit data peer-to-peer.

To enable real time communication in a web browser, three different components are needed:
\begin{enumerate}
\item An API useable on the client side inside the browser to both display and transmit media streams
\item Functions to retrieve media streams from devices like camera or microphone and pass them on to the browser for further use
\item Protocols and protocol stacks to ensure full duplex real time transmission
\end{enumerate}

\section{Maintained by the W3C WebRTC Working Group}
\subsection{Browser API}
To specify a proposed standard of the ECMAScript APIs for real time communications, the W3C WebRTC (Web Real-Time Communications) Working Group is working on the document "WebRTC 1.0: Real-time Communication Between Browsers".

These standards will be implemented in all major browser in the future

All information about the W3C efforts can be found at \url{http://www.w3.org/2011/04/webrtc/}.
It contains
\begin{itemize}
\item API functions for encoding and processing (e.g. filters) of media streams
\item API functions for establishing direct peer-to-peer connections, including firewall/NAT traversal
\item API functions for decoding and processing media streams at the incoming end
\item Delivery to the user of media streams, meaning displaying video and playing audio
\end{itemize}

Source: Charter, Working Draft
Does not contain
\begin{itemize}
\item Network protocols to establish the connections between peers (protocol considerations are worked on by the IETF)
\item Codecs for audio and video
\end{itemize}

Source: Charter

\section{Maintained by the IETF RTCWEB Work Group}


\todo{}