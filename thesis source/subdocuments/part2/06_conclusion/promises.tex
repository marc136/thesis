\documentclass[../../../thesis.tex]{subfiles}
%set currfolder to allow main file to compile and also this file to compile (\def does not work due to expansion issues)
\ifcurrfiledir{}{\edef\currfolder{.}}{\edef\currfolder{\currfiledir}}
\begin{document}

\chapter{Lessons learned}

\section{Promises}
Promises are a part of the coming ECMAScript 2015 standard\cite{es6-promises} to allow developers to implement data flow differently from using callback functions for dependency injection.\\

\subsection{Example with promises}
Listing~\ref{lst:example-promise.js} shows code that is used similarily in simucos to connect a user's WebRTC peer connection to a WebRTC endpoint on the media server and uses the kurento-client\cite{kurento-client-doc} JavaScript module.\par

Several functions need to be called after another while executing them asynchronously. \\
First, the user's WebRTC endpoint needs to be connected to the media server, then the recording of the video stream is started and in the end the user receives a message containing the SDP answer to actually start sending data to the media server.

\code{example-promise.js}{caption=Example for promises}


On line 2 in listing~\ref{lst:connect-promise.js} the Q library\cite{q-library} is used to create a new promise object. On line 7 an asynchronous call using an error-first callback function to retreive or create a media pipeline on the media server is started. Before this function is called on line 8, the promise object is returned on line 33.\par
When the callback function is called, a test is used to check for errors during that function. If no errors occur, the promise object is resolved on line 24 and returns the SDP answer. \par
If an error occurred a new exception is thrown in the context of the \texttt{connectToMediaServer} function that is caught in line 33 and rejects the promise object.\par
If the promise object was resolved, the function proceeds to line 4 of listing~\ref{lst:example-promise.js} and continues the asynchronous function calls.\par
If the promise object was rejected, or any other promises like \texttt{recordConnection} or \texttt{sendMessageTo} are rejected, the \texttt{fail} method is invoked in line 6 of listing~\ref{lst:example-promise.js} and centralized error handling can be used there.

\code{connect-promise.js}{caption=Example to return a promise object from callbacks}

\subsection{Example with callback functions}
The same example function can also be written using callback functions and would look like listing~\ref{lst:example-callback.js} to inject the functions into the callbacks.

\code{example-callback.js}{caption=Example for error-first callbacks}

The function \texttt{connectToMediaServer} from listing~\ref{lst:connect-promise.js} would not change much as seen in listing~\ref{lst:connect-callback.js}, but it would not return a promise object and isntead call a callback function that is passed into it.

\code{connect-callback.js}{caption=Example for error-first callbacks}


\subsection{Benefits from using promises}
These simple examples show the beauty of the cleaner syntax that is possible using a promise based API when compared to one based on callback functions.\\

WebRTC will also benefit from this and will allow even faster development cycles while simplifying the code structure and improving its readability.


\subsection{Combining the promise and error-first-callback approach}
The Q library\cite{q-library} also allows to create functions or modules that can both return a promise object and call a callback function.\\
This approach was actually used in simucos to allow other modules to keep using their approach when using the simucos module.\\

A simple example for this functionality is provided in listing~\ref{lst:promise-nodeify.js}, where the Node.js File API is used to read a configuration file from disk.

\code{promise-nodeify.js}{caption=Dual module that supports promises and callbacks}

\noindent
The function may then be used either with a promise-based approach as in listing~\ref{lst:promise-nodeify-promise.js} or using a callback-based approach as in listing~\ref{lst:promise-nodeify-callback.js}.

\code{promise-nodeify-promise.js}{caption=Using a dual module with promises}

\code{promise-nodeify-callback.js}{caption=Using a dual module with callbacks}


\end{document}