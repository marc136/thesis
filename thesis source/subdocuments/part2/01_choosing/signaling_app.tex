\documentclass[../../../thesis.tex]{subfiles}
%set currfolder to allow main file to compile and also this file to compile (\def does not work due to expansion issues)
\ifcurrfiledir{}{\edef\currfolder{.}}{\edef\currfolder{\currfiledir}}
\begin{document}

\section{Signaling application}

The signaling application was written for Node.js\footnote{Node.js\cite{node-js} is an application runtime and web server.} and uses the following node modules and JavaScript components.

\begin{table}[htbp]
\centering
\caption{Third party code used in the signaling application}
\begin{tabular}{l l l}
    \toprule
    \textbf{Component} & \textbf{License} & \textbf{Description}\\
    \midrule
    
    Adapter.js\footnotemark[3] & Apache 2.0 & Unifies the browser interfaces for WebRTC functions\\ 
    Bower\footnotemark[4] &	MIT & A package manager for Node.js and the web\\
    Express\footnotemark[5]	& MIT & Web framework for node.js - used to serve files to client \\
    Gulp\footnotemark[6] & MIT & Build tool to minify code and styles for production\\
    Kurento-client\footnotemark[7] & LGPL & Needed for RPCs to the Kurento media server \\
	Q\footnotemark[8] &	MIT & Library for promises\\
    Socket.io\footnotemark[9] & MIT & WebSocket server and client\\

		\bottomrule
\end{tabular}%
\label{tab:node-modules}%
\end{table}%

\footnotetext[3]{Adapter.js: \url{https://github.com/Temasys/AdapterJS}}
\footnotetext[4]{Bower: \url{http://bower.io/}}
\footnotetext[5]{Express: \url{https://github.com/strongloop/express}}
\footnotetext[6]{Gulp: \url{https://github.com/gulpjs/gulp}}
\footnotetext[7]{Kurento-client: \url{https://github.com/Kurento/kurento-client-js}}
\footnotetext[8]{Q: \url{https://github.com/kriskowal/q}}
\footnotetext[9]{Socket.io: \url{http://socket.io}}

\end{document}