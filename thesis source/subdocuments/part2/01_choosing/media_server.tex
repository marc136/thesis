\documentclass[../../../thesis.tex]{subfiles}
%set currfolder to allow main file to compile and also this file to compile (\def does not work due to expansion issues)
\ifcurrfiledir{}{\edef\currfolder{.}}{\edef\currfolder{\currfiledir}}
\begin{document}

\section{Media server}
Main focus of this thesis is to find out how multipoint conferencing with recording can be achieved for multiple participants using WebRTC.

That is why the implementation of a new media server is out of the scope of this thesis and it was decided at the beginning to evaluate existing media server solutions and use one.

\subsection{Comparison of existing media server solutions}
Many different possibilities were evaluated and a short listing of that evaluation is located in the appendix on page~\pageref{chap:compare-media-servers}.\\
Most of those solutions may be deployed for self-hosting, but also a number of cloud based hosted services were investigated and documented.

\subsection{Selection of Kurento media server}
After careful consideration, the \textbf{Kurento media server}\cite{kurento} was chosen because it supports transcoding between the two required codecs H.264 and VP8\cite{ietf-rtcweb-video}, and is released under the permissive LGPL license that allows to use it unaltered in a commercial environment.\par
It also supports the concept of pluggable media pipelines that allows the implementation of both mixed and relayed media servers\footnotemark, and extensive documentation.
\footnotetext{Mixed and relayed media servers are defined in section~\ref{sec:multipoint-architectures} starting on page~\pageref{subsec:multipoint-architectures}}\par
Also a JavaScript API and interface implementation is included, that allows the same functionality as connecting to the media server from a Java Application Server, without needing to implement a new wrapper for remote procedure calls.\\

The Kurento media server can be built on any Linux machine and a pre-compiled Debian package of the stable version is also available, which makes it very easy to set up.\\

In the programmed prototype, the media server and signaling application are running on the same machine. The media server is started as a system service and the signaling application is started as a Node.js application.

\end{document}